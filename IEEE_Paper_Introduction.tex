\section{Introduction}

Contracts are fundamental to business transactions, legal agreements, and organizational operations. They define obligations, responsibilities, and legal protections between parties. However, the process of drafting, reviewing, and understanding contracts remains highly manual, time-consuming, and prone to errors. Small loopholes, unclear language, or missing clauses can result in misinterpretations, financial losses, or legal disputes. In traditional legal environments, contract creation and analysis are handled separately, often requiring expertise from legal professionals. For individuals, startups, or organizations with limited access to legal counsel, this can be a major barrier.

With the advancement of Natural Language Processing (NLP) and transformer-based models, AI-driven contract tools have begun to emerge. Yet, most of the existing platforms are limited to narrow functionalities: some focus solely on clause extraction or risk detection, while others assist only with drafting predefined templates. Very few offer a complete system that supports both contract analysis and creation, and almost none provide built-in educational components to help users understand the legal context of their contracts.

To address this gap, we propose a unified AI-powered contract platform that combines three core modules: contract analysis, contract creation, and legal education. Our platform leverages a fine-tuned RoBERTa model for clause-level analysis and risk scoring, integrated with a Groq AI-backed generation system for intelligent drafting. The educational module complements these functionalities by offering real-time explanations, contract-type breakdowns, and risk visualizations, making the platform accessible not only to professionals but also to students and non-experts.

The Contract Analysis Module includes features such as clause extraction, loophole detection, batch document analysis, and side-by-side contract comparisons. It is trained using the CUAD dataset, covering over 40 legal categories. The Contract Creation Module supports voice-enabled inputs, customizable contract templates, clause recommendations, and generation in multiple contract formats such as service agreements, employment contracts, and vendor terms. Meanwhile, the Legal Education and Dashboard Module provides simplified legal terminology, a visual dashboard for risk insights, and contract-type tutorials to enhance user understanding.

To the best of our knowledge, this is the first AI platform to unify contract analysis, creation, and legal education in a single, accessible system. Our contributions include: (1) a novel dual-AI architecture combining fine-tuned RoBERTa for understanding with large language models for generation, (2) an integrated educational framework that makes legal knowledge accessible to non-experts, (3) comprehensive evaluation demonstrating significant improvements in processing speed and accuracy over traditional methods, and (4) a scalable platform design supporting both individual users and enterprise-level batch processing requirements.

The remainder of this paper is organized as follows: Section II presents related work in legal NLP and contract analysis. Section III describes our system architecture and the three core modules. Section IV details the implementation methodology and technical approaches. Section V presents experimental results and user studies. Section VI discusses limitations and future work, and Section VII concludes the paper.
