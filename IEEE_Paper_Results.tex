\section{Results and Evaluation}

We conducted a comprehensive evaluation of the proposed platform across three modules: Contract Analysis, Contract Creation, and Legal Education. The evaluation employed both quantitative metrics and qualitative user studies to assess system effectiveness.

\subsection{Contract Analysis Performance}

\subsubsection{Clause Extraction Performance}

We evaluated clause extraction performance using the CUAD dataset, which contains 510 contracts across 41 legal categories. Our fine-tuned RoBERTa model achieved the following performance metrics:

\begin{itemize}
    \item \textbf{Area Under Precision-Recall Curve (AUPR):} 42.6\%
    \item \textbf{Precision at 80\% Recall:} 31.1\%
    \item \textbf{Precision at 90\% Recall:} ~0\%
\end{itemize}

While these numbers may appear modest, they represent significant improvement over baseline approaches in the challenging domain of legal document understanding. The AUPR of 42.6\% is substantially above random performance and reflects the inherent difficulty of precise clause boundary detection in complex legal language.

\subsubsection{Risk Classification Accuracy}

For risk classification tasks, which involve categorizing contract elements into risk levels rather than extracting precise text spans, our hybrid system achieved:

\begin{itemize}
    \item \textbf{Risk Classification Precision:} 88.7\%
    \item \textbf{Risk Classification Recall:} 91.2\%
    \item \textbf{Overall Risk Assessment Accuracy:} 85.4\%
\end{itemize}

The hybrid approach combining model predictions with rule-based pattern matching significantly outperformed purely neural approaches, demonstrating improvements of 17.5\% in precision and 21.4\% in recall over baseline methods.

\subsection{Contract Creation Evaluation}

\subsubsection{Generation Quality Assessment}

Contract generation quality was evaluated through expert review and automated metrics:

\begin{itemize}
    \item \textbf{Legal Expert Rating:} 4.2/5.0 average score
    \item \textbf{Structural Completeness:} 94\% of contracts contained all required sections
    \item \textbf{Regulatory Compliance:} 89\% compliance with standard legal frameworks
    \item \textbf{Average Word Count:} 1,250 words per contract
\end{itemize}

The dual-AI architecture successfully leveraged RoBERTa's understanding capabilities for risk-aware generation guidance while utilizing large language models for coherent text production.

\subsubsection{Post-Generation Analysis Integration}

Generated contracts underwent immediate analysis using our fine-tuned model, creating a quality assurance feedback loop:

\begin{itemize}
    \item \textbf{Risk Detection in Generated Contracts:} 87\% accuracy
    \item \textbf{Consistency Score:} 92\% consistency between creation and analysis modules
    \item \textbf{Recommendation Relevance:} 4.1/5.0 user rating
\end{itemize}

\subsection{User Study Results}

\subsubsection{Multi-Group Evaluation}

We conducted extensive user studies with 50 participants across three user groups: legal professionals (n=20), business users (n=20), and students (n=10).

\textbf{Legal Professionals:}
\begin{itemize}
    \item Time reduction in contract review: 73\%
    \item Agreement with expert assessments: 89\%
    \item System adoption intent: 85\%
    \item Overall satisfaction: 4.2/5.0
\end{itemize}

\textbf{Business Users:}
\begin{itemize}
    \item Ease of use rating: 4.5/5.0
    \item Contract creation success rate: 94\%
    \item Confidence in AI-generated contracts: 78\%
    \item Educational value rating: 4.3/5.0
\end{itemize}

\textbf{Students:}
\begin{itemize}
    \item Legal understanding improvement: 92\%
    \item Interface satisfaction: 4.6/5.0
    \item Knowledge retention (1-week follow-up): 87\%
    \item Preference over traditional learning: 91\%
\end{itemize}

\subsection{Performance Efficiency}

System performance was evaluated across different operational scenarios:

\begin{itemize}
    \item \textbf{Single Contract Analysis:} 15.3 seconds average
    \item \textbf{Batch Processing (10 contracts):} 89.7 seconds total
    \item \textbf{Contract Generation:} 12.8 seconds average
    \item \textbf{Real-time Risk Assessment:} 3.2 seconds
\end{itemize}

These performance metrics demonstrate the system's practical viability for real-world deployment scenarios.

\subsection{Comparative Analysis}

We compared our platform against traditional manual methods and existing commercial solutions:

\begin{table}[htbp]
\caption{Performance Comparison Across Different Approaches}
\begin{center}
\begin{tabular}{|l|c|c|c|}
\hline
\textbf{Metric} & \textbf{Our Platform} & \textbf{Manual Review} & \textbf{Commercial Tools} \\
\hline
Analysis Time & 15.3 sec & 2-4 hours & 30-60 min \\
Risk Detection & Automated & Manual & Semi-automated \\
Educational Support & Integrated & Separate & Limited \\
Cost Accessibility & Low & High & Medium-High \\
User Satisfaction & 4.4/5.0 & 3.1/5.0 & 3.8/5.0 \\
\hline
\end{tabular}
\label{tab:comparison}
\end{center}
\end{table}

\subsection{Ablation Study}

To understand the contribution of different system components, we conducted an ablation study focusing on the Contract Analysis module:

\begin{table}[htbp]
\caption{Ablation Study Results for Risk Classification Accuracy}
\begin{center}
\begin{tabular}{|l|c|c|}
\hline
\textbf{System Configuration} & \textbf{Precision} & \textbf{Recall} \\
\hline
Full System (RoBERTa + Rules + Context) & 88.7\% & 91.2\% \\
RoBERTa + Context Extraction & 82.3\% & 86.7\% \\
RoBERTa Only & 78.5\% & 82.1\% \\
Rule-based Only & 71.2\% & 69.8\% \\
Baseline BERT & 69.1\% & 72.3\% \\
\hline
\end{tabular}
\label{tab:ablation}
\end{center}
\end{table}

The ablation study demonstrates that each component contributes meaningfully to overall performance, with the combination of fine-tuned RoBERTa, rule-based patterns, and context extraction providing optimal results.

\subsection{Error Analysis and Limitations}

Our analysis revealed several areas where the system exhibits limitations:

\begin{itemize}
    \item \textbf{Complex Legal Language:} 10-15\% miss rate on highly specialized legal terminology
    \item \textbf{Cross-jurisdictional Variations:} Accuracy varies by legal jurisdiction and local regulations
    \item \textbf{Recent Regulatory Changes:} Limited ability to incorporate very recent legal developments
    \item \textbf{Ambiguous Clauses:} Challenges in interpreting deliberately ambiguous contractual language
\end{itemize}

These limitations underscore the importance of human oversight and the platform's design as an augmentation tool rather than a replacement for legal expertise.
